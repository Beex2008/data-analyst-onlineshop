% Anhang: nicht nummerierter Teil
\section*{Anhang}
\addcontentsline{toc}{section}{Anhang}

\subsection*{Spaltenumbenennungen der CSV-Daten}
\addcontentsline{toc}{subsection}{Spaltenumbenennungen der CSV-Daten}

Die Spalten des ursprünglichen Datensatzes wurden vor der weiteren Verarbeitung umbenannt; die Quelldaten aus dem Datensatz \cite{online_shoppers_purchasing_intention_dataset_468} haben uns bei den Spaltenumbenennungen als Orientierung gedient. Eine einheitliche Benennung nach dem Schema \texttt{snake\_case}\footnote{\url{https://developer.mozilla.org/de/docs/Glossary/Snake_case}} erleichtert die Lesbarkeit, die konsistente Weiterverarbeitung und die Referenzierung im Bericht sowie im Dashboard. Die vollständige Übersicht aller Umbenennungen ist in Tabelle~\ref{tab:spaltenumbenennung} dargestellt.

\begin{longtable}{p{4cm}p{4.5cm}p{6.5cm}}
\caption{Spaltenumbenennungen der CSV-Daten}\label{tab:spaltenumbenennung} \\
\toprule
\textbf{Original CSV-Spalte} & \textbf{Neuer Spaltenname} & \textbf{Begründung} \\
\midrule
\endfirsthead
\caption{Spaltenumbenennungen der CSV-Daten (Fortsetzung)} \\
\toprule
\textbf{Original CSV-Spalte} & \textbf{Neuer Spaltenname} & \textbf{Begründung} \\
\midrule
\endhead
\multicolumn{3}{c}{{\bfseries \tablename\ \thetable{} -- Fortsetzung}} \\
\toprule
\textbf{Original CSV-Spalte} & \textbf{Neuer Spaltenname} & \textbf{Begründung} \\
\midrule
\endhead
\bottomrule
\endfoot
\bottomrule
\endlastfoot
Administrative & \texttt{admin\_page\_count} & Einheitliche Benennung für Seitenanzahlen (\_page\_count), beschreibt die Anzahl administrativer Seitenaufrufe pro Session. \\
Administrative\_Duration & \texttt{admin\_page\_duration} & Einheitliche Benennung für Zeitangaben (\_page\_duration), gibt die Verweildauer auf administrativen Seiten an. \\
Informational & \texttt{info\_page\_count} & Vereinheitlichung der Seitenkategorien, beschreibt die Anzahl besuchter Informationsseiten. \\
Informational\_Duration & \texttt{info\_page\_duration} & Klare Trennung zwischen Seitenanzahl und Verweildauer auf Informationsseiten. \\
ProductRelated & \texttt{product\_page\_count} & Zentrale Variable zur Überprüfung der Hypothese H1, da Produktseiten unmittelbar mit Kaufentscheidungen zusammenhängen. \\
ProductRelated\_Duration & \texttt{product\_page\_duration} & Ergänzt product\_page\_count um die zeitliche Intensität der Produktbetrachtung. \\
BounceRates & \texttt{bounce\_rate} & Vereinfachter Name für eine bekannte Google-Analytics-Metrik, beschreibt Absprungraten pro Seite. \\
ExitRates & \texttt{exit\_rate} & Vereinfachter, einheitlicher Name für eine weitere Google-Analytics-Metrik, beschreibt Ausstiegsraten. \\
PageValues & \texttt{page\_value} & Kürzere, besser interpretierbare Bezeichnung für den monetären Wert einer Seite vor einem Kauf. \\
SpecialDay & \texttt{special\_day\_score} & Verdeutlicht, dass es sich um einen normierten Score (0--1) handelt, der die Nähe zu Aktionstagen abbildet. \\
Month & \texttt{visit\_month} & Präzisiert die Bedeutung als Monat der Session, nicht als abstrakter Kalendermonat. \\
OperatingSystems & \texttt{operating\_system\_id} & Technische Kontextvariable, numerische Kodierung zur Identifikation des Betriebssystems. \\
Browser & \texttt{browser\_id} & Technische Kontextvariable, numerische Kodierung des verwendeten Browsers. \\
Region & \texttt{region\_id} & Regionale Zuordnung des Nutzers, technisch als ID gespeichert. \\
TrafficType & \texttt{traffic\_type\_id} & Kennzeichnet den Ursprung des Traffics (z.\,B. Direkt, Referral, Kampagne). \\
VisitorType & \texttt{visitor\_type} & Beibehaltung der semantischen Bedeutung (``New'' / ``Returning''), nur sprachlich vereinheitlicht. \\
Weekend & \texttt{weekend\_flag} & Boolean-Indikator, der klar als Ja/Nein-Merkmal erkennbar ist. \\
Revenue & \texttt{revenue\_flag} & Kennzeichnet, ob ein Kauf stattgefunden hat (TRUE/FALSE); Umbenennung macht die Rolle als Zielvariable eindeutig. \\
\end{longtable}

