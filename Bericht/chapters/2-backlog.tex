% 2. Backlog und Projektziele (IHK-Struktur)
\section{Backlog und Projektziele}

\subsection{Sammlung von User Storys und Potenzialen}
Auf Basis der Daten wurden erste User Storys erstellt, die uns helfen, Fragestellungen aus Sicht des Unternehmens zu formulieren. Die Projektplanung und -verwaltung wurde mithilfe eines Kanban-Boards (siehe \cref{fig:backlogs}) und eines Gantt-Diagramms (siehe \cref{fig:gantt}) durchgeführt; beide Visualisierungen befinden sich in Kapitel~\ref{sec:anhang}.

\begin{itemize}[leftmargin=*]
	\item \textbf{US\,1 -- Produktinteraktion beeinflusst Kaufabschluss positiv}\\
	Als Online-Marketing-Manager: Möchte ich vorhersagen, ob eine Session zu einem Kauf führt, um Marketingmaßnahmen gezielt auszurichten.

	\item \textbf{US\,2 -- Returning Visitors kaufen häufiger als neue Besucher}\\
	Als Marketing Manager: Möchte ich zwischen neuen und wiederkehrenden Besuchern unterscheiden können, um gezielte Marketingmaßnahmen für Returning Visitors umzusetzen.

	\item \textbf{US\,3 -- Je länger sich der Besucher auf der Seite aufhält, desto wahrscheinlicher der Kaufabschluss}\\
	Als Data Analyst: Möchte ich Muster im Nutzerverhalten erkennen, um wichtige Einflussfaktoren des Kaufabschlusses zu identifizieren.

	\item \textbf{US\,4 -- Besuche an Wochenenden führen zu höherer Kaufwahrscheinlichkeit}\\
	Als Produktmanager: Möchte ich ein ML-Modell, das die Kaufwahrscheinlichkeit prognostiziert, um Nutzer in Echtzeit zu klassifizieren.

	\item \textbf{US\,5 -- Returning Visitors haben eine höhere Conversion Rate als New Visitors}\\
	Als Marketing Manager: Möchte ich vorhersagen, ob eine Session zu einem Kauf führt, um Marketingmaßnahmen gezielt auszurichten.
\end{itemize}

\subsection{Bewertung und Priorisierung nach Aufwand und Ertrag (Low Hanging Fruits)}
Nach der Sammlung der verschiedenen Ansätze wurde die erste User Story priorisiert. Die Fragestellung liefert einen direkten Einblick in das Kaufverhalten der Nutzer; die Kennzahlen Page Count und Page Duration stehen in engem Zusammenhang mit dem Kaufabschluss und eignen sich gut, um das Kundeninteresse während einer Session einzuschätzen. Die Priorisierung folgte dem Prinzip der Low Hanging Fruits: Es wurden Fragen ausgewählt, die mit geringem Aufwand untersucht werden können und einen hohen Mehrwert bieten.

\subsection{Auswahl der erreichbaren Projektziele für den ersten Sprint}
Für den ersten Sprint wurden Punkte ausgewählt, die mit geringem Aufwand umsetzbar sind und die vorhandenen Daten direkt nutzen. Ziel ist es, mögliche Zusammenhänge, ein besseres Verständnis und erste Ergebnisse bis zum Ende des ersten Sprints sichtbar zu machen – ohne dass dafür z.\,B. KNIME oder Power BI zwingend nötig sind. Konkret: Fokus auf Datenaufbereitung, EDA, Modellierung (Kaufvorhersage) sowie Visualisierung und Reporting in Power BI.
