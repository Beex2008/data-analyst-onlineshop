\section{Einleitung}

\subsection{Motivation und Zielsetzung des Projekts}
% Hier den Text für Motivation und Zielsetzung einfügen

\subsection{Ausgangssituation und Hintergrund der Daten}
Die vorliegenden Daten stammen aus dem Webtracking eines Online-Shops und erfassen das Verhalten von Besuchern während einzelner Website-Sessions. Ziel des Unternehmens ist es, besser zu verstehen, unter welchen Bedingungen ein Kaufabschluss erfolgt, um Marketing- und Shop-Maßnahmen gezielt zu optimieren.

Die Daten bilden einen realistischen E-Commerce-Prozess ab (vergleichbar mit Google Analytics) und enthalten Informationen zu:
\begin{itemize}
	\item Nutzerverhalten (z. B. Produktseiten, Verweildauer),
	\item Einkaufssituation (Monat, Wochenende, Special Days),
	\item technischem Kontext (Browser, Betriebssystem),
	\item sowie dem Kaufabschluss (Revenue).
\end{itemize}

\subsection{Geschäftlicher Hintergrund (Online-Shop \& Kaufabschlüsse)}
% Hier den Text für den geschäftlichen Hintergrund einfügen

\subsection{Projektabgrenzung und Fokus der Analyse}
% Hier den Text für Projektabgrenzung und Fokus einfügen

\subsection{Hypothese H1: Zusammenhang zwischen Produktinteraktion und Kaufabschluss}
% Hier den Text für Hypothese H1 einfügen
