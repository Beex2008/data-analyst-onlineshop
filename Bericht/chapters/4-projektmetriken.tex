% 4. Projektmetriken (IHK-Struktur)
\section{Projektmetriken}

\subsection{Fortschritt und Erfolg des Projektes sowie Qualität der Analyse bzw. des Dashboards im Einsatz}
Für die weitere Analyse werden zunächst relevante Attribute ausgewählt, die einen Einfluss auf den Kaufabschluss haben. Dazu zählen unter anderem das Nutzerverhalten auf Produktseiten, der Besuchertyp sowie zeitliche Merkmale.

Die Daten werden in KNIME aufbereitet. Dabei ist geplant, Ausreißer zu identifizieren und gegebenenfalls zu bereinigen, um verzerrte Ergebnisse zu vermeiden. Anschließend werden die Daten in Power BI visualisiert, um Zusammenhänge zu erkennen.

Relevante Attribute der Hypothese: Nutzerverhalten auf Produktseiten; Besuchertyp; zeitliche Merkmale (z.\,B. Wochenende, Monat).

Zusätzlich lassen sich Fortschritt und Erfolg messen durch: Erfüllung der Sprintziele (Backlog-Status), Qualität des Modells (z.\,B. Vorhersagegüte), Nutzbarkeit des Dashboards für Fachabteilungen sowie die Definition und Einhaltung von Business-KPIs (vgl. Kapitel~8). Die Qualität der Analyse zeigt sich in der Nachvollziehbarkeit der EDA, der Datenqualität und der Erklärbarkeit des Modells.

\subsection{Definitions of done für die Funktionen und die Mehrwerte der ausgewählten User Stories}
\textbf{User Story 1:} Als Betreiber eines Online-Shops möchte ich verstehen, wie sich die Anzahl der Seitenaufrufe und die Verweildauer auf Produktseiten auf den Kaufabschluss auswirken.

\textbf{Definition of Done:}
\begin{itemize}
	\item Die relevanten Variablen (Page Count, Page Duration, Revenue) wurden eindeutig identifiziert.
	\item Es liegt eine nachvollziehbare Auswertung vor, die zeigt, ob und wie sich Seitenaufrufe und Verweildauer auf den Kaufabschluss auswirken.
	\item Die Ergebnisse sind so aufbereitet, dass sie auch für fachliche Stakeholder verständlich sind.
\end{itemize}

\textbf{Mehrwert:} Das Unternehmen erhält ein besseres Verständnis dafür, welche Nutzerinteraktionen kaufrelevant sind. Marketing- und Produktmaßnahmen können gezielter auf kaufinteressierte Nutzer ausgerichtet werden. Erste Optimierungspotenziale im Online-Shop können identifiziert werden, ohne großen technischen Aufwand.

Zusätzlich gelten die übergreifenden DoD: Daten aufbereitet und dokumentiert; EDA durchgeführt und bewertet; Modell trainiert und evaluiert; Dashboard in Power BI erstellt und mit KNIME-Daten angebunden; Dokumentation und Anhang (Kanban, Gantt) vollständig.
