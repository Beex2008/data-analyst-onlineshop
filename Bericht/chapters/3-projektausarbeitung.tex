% 3. Projektausarbeitung (IHK-Struktur)
\section{Projektausarbeitung}

\subsection{Formulierung von Hypothesen, die die Machbarkeit der Funktionen der User Stories untermauern}
Für jede User Story wurden Hypothesen formuliert, um die Machbarkeit der geplanten Maßnahmen anhand der vorhandenen Daten zu untermauern.

\textbf{H1 -- Produktinteraktion beeinflusst Kaufabschluss positiv:} Nutzer, die mehr produktbezogene Seiten ansehen oder mehr Zeit darauf verbringen, schließen Käufe ab. (Var: Product-related und Revenue.) Diese Hypothese untermauert die Machbarkeit der User Story zur Kaufvorhersage und wurde in der EDA (Kapitel~1) sowie in der Modellierung (Kapitel~6) überprüft.

\textbf{H2 -- Returning Visitors kaufen häufiger als neue Besucher.} Var: Visitor Type und Conversion-Rate (Anzahl Revenue=True / Anzahl der Sessions).

\textbf{H3 -- Je länger sich der Besucher auf der Seite aufhält, desto wahrscheinlicher der Kaufabschluss.} Var: Product-related-Duration und Revenue.

\textbf{H4 -- Besuche an Wochenenden führen zu höherer Kaufwahrscheinlichkeit.} Var: Conversion-Rate und Weekend.

\textbf{H5 -- Returning Visitors haben eine höhere Conversion Rate als New Visitors.} Messvariable: \texttt{visitor\_type}, \texttt{revenue\_flag} (relevant für Marketingmanager).

\subsection{Zuordnung der Hypothesen zu User Stories und Potenzialen}
Diese Hypothesen verdeutlichen, welche Zusammenhänge im Nutzerverhalten untersucht werden sollen. Sie zeigen, welche Annahmen den jeweiligen Fragestellungen zugrunde liegen und welches Optimierungspotenzial besteht – und machen auf einen Blick erkennbar, welche Maßnahme sinnvoll ist und welche Kundengruppe betroffen ist.

\begin{table}[H]
\centering
\caption{Zuordnung der User Stories zu Hypothesen und Potenzialen}
\label{tab:hypothesen-zuordnung}
\small
\begin{tabular}{p{4.2cm}p{3.8cm}p{5.2cm}}
\toprule
\textbf{User Story} & \textbf{Potenzial} & \textbf{Hypothese} \\
\midrule
Als Online-Marketing-Manager: vorhersagen, ob eine Session zu einem Kauf führt, um Marketingmaßnahmen gezielt auszurichten. & Optimierung von Produktseiten, bessere Darstellung und Hervorhebung & Nutzer, die mehr auf produktbezogene Seiten interagieren oder mehr Zeit darauf verbringen, schließen Käufe ab. Var: Product-related und Revenue. \\
\midrule
Als Marketing Manager: zwischen neuen und wiederkehrenden Besuchern unterscheiden, gezielte Maßnahmen für Returning Visitors. & Stärkung von Kundenbindung, gezieltes Retargeting und Newsletter-Maßnahmen & Returning Visitors haben eine höhere Conversion Rate als neue Besucher. Var: Visitor Type und Conversion-Rate. \\
\midrule
Als Data Analyst: Muster im Nutzerverhalten erkennen, Einflussfaktoren des Kaufabschlusses identifizieren. & Verbesserung der Nutzerführung, Reduktion von Absprüngen, optimierte Seitenstruktur & Je länger sich der Besucher auf der Seite aufhält, desto wahrscheinlicher der Kaufabschluss. Var: Product-related-Duration und Revenue. \\
\midrule
Als Produktmanager: ML-Modell zur Prognose der Kaufwahrscheinlichkeit, Nutzer in Echtzeit klassifizieren. & Zeitlich optimierte Marketingkampagnen und Angebote am Wochenende & Besuche an Wochenenden führen zu höherer Kaufwahrscheinlichkeit. Var: Conversion-Rate und Weekend. \\
\midrule
Als Marketing Manager: vorhersagen, ob eine Session zu einem Kauf führt, Marketingmaßnahmen ausrichten. & Effizienter Einsatz des Marketingbudgets, Fokus auf profitable Zielgruppen & Returning Visitors haben eine höhere Conversion Rate als New Visitors. Messvariable: \texttt{visitor\_type}, \texttt{revenue\_flag}. \\
\bottomrule
\end{tabular}
\end{table}
