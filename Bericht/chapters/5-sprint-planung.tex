% 5. Planung des ersten Sprints (IHK-Struktur)
\section{Planung des ersten Sprints}

\subsection{Projektorganisation und Kommunikation}
Die Zusammenarbeit im Projektteam wurde durch klar definierte Kommunikationswege und ein zentrales Projektmanagement unterstützt. Die Abstimmung erfolgte regelmäßig über Zoom und WhatsApp und ermöglichte eine flexible, ortsunabhängige Kommunikation. Die Besprechungen dienten der inhaltlichen Abstimmung, der Fortschrittskontrolle und gemeinsamen Entscheidungen.

Die Datenbasis wird wöchentlich aktualisiert, sodass Analysen und Auswertungen stets auf dem aktuellen Stand sind. Durch die regelmäßigen Updates können Hypothesen strukturiert überprüft werden.

Zur Planung, Koordination und Dokumentation der Aufgaben kam Trello als Projektmanagement-Tool zum Einsatz, da es eine übersichtliche Strukturierung der Aufgaben ermöglicht. Dort wurden alle User Stories, Arbeitspakete und Aufgaben erfasst, priorisiert und den jeweiligen Projektphasen zugeordnet. So konnten die einzelnen Aufgabenpakete klar definiert werden; der Fortschritt der Arbeit war jederzeit ersichtlich.

\subsection{Formulierung der einzelnen Aufgabenpakete mit ihren berechneten oder geschätzten Aufwänden}
Die Aufgabenpakete wurden im Kanban-Board (siehe \cref{fig:backlogs}) und im Gantt-Diagramm (siehe \cref{fig:gantt}) in Kapitel~8 abgebildet. Die folgenden Tabellen fassen zentrale Pakete und geschätzte Aufwände zusammen.

\begin{table}[H]
\centering
\caption{Erstes Aufgabenpaket: Datenüberprüfung}
\label{tab:aufwand-datenpruefung}
\begin{tabular}{p{3.2cm}p{7cm}p{2.8cm}}
\toprule
\textbf{Aufgabenpaket} & \textbf{Beschreibung} & \textbf{Geschätzter Aufwand} \\
\midrule
Datenüberprüfung & Überprüfung der Vollständigkeit und Plausibilität der vorhandenen Kundendaten & ca.\,1,5 Stunden \\
\bottomrule
\end{tabular}
\end{table}

\begin{table}[H]
\centering
\caption{Weitere Aufgabenpakete des ersten Sprints}
\label{tab:aufwand-sprint}
\begin{tabular}{p{3.8cm}p{6.8cm}p{2.6cm}}
\toprule
\textbf{Aufgabenpaket} & \textbf{Beschreibung} & \textbf{Geschätzter Aufwand} \\
\midrule
Einfache Datenaufbereitung & Kategorisierung der Kundendaten nach Besuchertyp, Nutzungsverhalten und Kaufstatus & ca.\,2 Stunden \\
Hypothesenprüfung & Erste Einschätzung, ob die Hypothesen plausibel sind & ca.\,1,5 Stunden \\
Erste Visualisierung & Darstellung der Kennzahlen in Diagrammen, um Trends erkennbar zu machen & ca.\,1 Stunde \\
Dokumentation der Ergebnisse & Zusammenfassung der Aufgaben, Beschreibung und Erkenntnisse & ca.\,1,5 Stunden \\
\bottomrule
\end{tabular}
\end{table}

Weitere zentrale Pakete: Projektvorbereitung und Ist-Analyse, Datenaufbereitung in KNIME, Spaltenumbenennen, EDA, Modellierung (XGBoost), Datenübergabe KNIME--Power BI, Business-KPIs und Dashboard-Design, Interpretation und Management-Mehrwert. Die Aufwände wurden im Gantt-Diagramm geplant und in den Sprint-Reviews überprüft. Weitere Darstellungen (Kanban-Board, Gantt-Diagramm) siehe Kapitel~8.
