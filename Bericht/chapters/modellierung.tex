\section{Modellierung und Vorhersage}
Für die Modellierung wurde bewusst ein begrenztes Feature-Set gewählt. Ziel war es nicht, möglichst viele Variablen zu verwenden, sondern diejenigen, die fachlich sinnvoll begründet und gut interpretierbar sind.
Im finalen Feature-Set enthalten sind:
\begin{itemize}
	\item \texttt{product\_page\_count} als zentrales Hypothesenmerkmal
	\item \texttt{product\_share} als Verstärkung des Produktfokus
	\item \texttt{product\_page\_duration} als Maß für Engagement
	\item \texttt{page\_value} als kaufnahe Metrik
	\item \texttt{total\_duration} zur Abbildung der Sessionintensität
	\item kodierte Kontextvariablen wie Monat, Besuchertyp und Wochenende
\end{itemize}

\subsection{Modellierung}
Für die Modellierung wurde das Gradient Boosting-Modell verwendet. 
\subsection{Vorhersage}
Die Vorhersage wurde mithilfe des trainierten Modells durchgeführt. Die Vorhersage wurde mithilfe der predict-Methode des Modells durchgeführt.
